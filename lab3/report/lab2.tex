\documentclass[a4paper,14pt]{extarticle}

\usepackage[utf8x]{inputenc}
\usepackage[T1,T2A]{fontenc}
\usepackage[russian]{babel}
\usepackage{hyperref}
\usepackage{indentfirst}
\usepackage{here}
\usepackage{array}
\usepackage{graphicx}
\usepackage{caption}
\usepackage{subcaption}
\usepackage{chngcntr}
\usepackage{amsmath}
\usepackage{amssymb}
\usepackage{pgfplots}
\usepackage{pgfplotstable}
\usepackage[left=2cm,right=2cm,top=2cm,bottom=2cm,bindingoffset=0cm]{geometry}
\usepackage{multicol}
\usepackage{askmaps}
\usepackage{titlesec}

\renewcommand{\le}{\ensuremath{\leqslant}}
\renewcommand{\leq}{\ensuremath{\leqslant}}
\renewcommand{\ge}{\ensuremath{\geqslant}}
\renewcommand{\geq}{\ensuremath{\geqslant}}
\renewcommand{\epsilon}{\ensuremath{\varepsilon}}
\renewcommand{\phi}{\ensuremath{\varphi}}
\renewcommand{\thefigure}{\arabic{figure}} 	
\renewcommand*\not[1]{\overline{#1}}

\titleformat*{\section}{\large\bfseries} 
\titleformat*{\subsection}{\normalsize\bfseries} 
\titleformat*{\subsubsection}{\normalsize\bfseries} 
\titleformat*{\paragraph}{\normalsize\bfseries} 
\titleformat*{\subparagraph}{\normalsize\bfseries} 

\counterwithin{figure}{section}
\counterwithin{equation}{section}
\counterwithin{table}{section}
\newcommand{\sign}[1][5cm]{\makebox[#1]{\hrulefill}}
\graphicspath{{../pics/}}
\captionsetup{justification=centering,margin=1cm}
\def\arraystretch{1.3}
\setlength\parindent{5ex}
\titlelabel{\thetitle.\quad}

\begin{document}

\begin{titlepage}
\begin{center}
	Санкт-Петербургский Политехнический Университет Петра Великого\\[3mm]
	Институт компьютерных наук и технологий \\[3mm]
	Кафедра компьютерных систем и программных технологий\\[6cm]
	
	\textbf{КУРСОВОЙ ПРОЕКТ}\\[3mm]
	\textbf{Разработка устройства передачи данных}\\[3mm]
	по дисциплине <<Автоматизация проектирования\\[3mm] 
	дискретных устройств>>\\[6cm]
\end{center}

\begin{flushleft}
	Выполнил\\[3mm]
	студент гр. 33501/4  \hspace*{7.1cm} Дьячков В.В.\\[3mm]
	Руководитель\\[3mm]
	доцент, к.т.н. \hspace*{8.5cm} А.С. Филиппов\\[4mm]
	\hspace*{11.5cm} <<\sign[7mm]>> \sign[20mm] \the\year\hspace{1mm} г.
\end{flushleft}

\vfill

\begin{center}
	Санкт-Петербург\\
	\the\year
\end{center}
\end{titlepage}

\addtocounter{page}{1}
\counterwithin{lstlisting}{section}

\tableofcontents
\listoffigures
\lstlistoflistings
\newpage

\section{Задачи работы}

\begin{enumerate}
\setlength\itemsep{0em}
\item Задание требований к тактовой частоте проекта.
\item Работа с приложением TimeQuest.
\item Анализ полученных результатов для максимальной тактовой частоты работы устройства.
\end{enumerate}

\section{Синтез счетчика}

\subsection{Результаты синтеза}

На рис. \ref{fig:counter} изображена синтезированная схема 4-разрядного счетчика.

\begin{figure}[H]
\begin{center}
	\includegraphics[width=\textwidth]{counter}
	\caption{Синтезированная схема}
	\label{fig:counter}
\end{center}
\end{figure}

\subsection{Результаты моделирования}

На рис. \ref{fig:func-modeling} изображен результат функционального моделирования, полученный с помощью встроенной системы моделирования – University Program VWF.

\begin{figure}[H]
\begin{center}
	\includegraphics[scale=0.9]{modeling}
	\caption{Результаты моделирования}
	\label{fig:func-modeling}
\end{center}
\end{figure}

%TODO Проведите анализ временной диаграммы и объясните почему:
%TODO Данные записываемые в счетчик появляются на выходе с задержкой 2 такта?
%TODO В самом начале моделирования значение 0 присутствует на выходе 2 такта?

\subsection{Временной анализ}

\subsubsection{Создание SDC файла}

В листинге \ref{code:sdc} приведена сформированная SDC команда, создающая ограничения для тактового сигнала.

\begin{lstlisting}[caption=Synopsys Design Constraints (SDC) файл, label=code:sdc]
create_clock -name input_clk -period 20.000 [get_ports {clk}]
\end{lstlisting}

В отчете компиляции в разделе \textbf{TimeQuest Timing Analyzer} указана максимальная тактовая частота работы проекта: Fmax = $465.55$ MHz, Restricted Fmax = $250$ MHz.

В таблице \textbf{Multicorner Timing Analysis Summary} приведены следующие значения задержек для худшего случая:
\begin{itemize}
\setlength\itemsep{0em}
\item Setup Slack = $17.852$ нс.
\item Hold Slack = $0.197$ нс.
\item Slack для Minimum Pulse Width = $9.436$ нс.
\end{itemize}

\subsubsection{Изменение периода тактового сигнала}

При изменении ограничения с 20 нс до 1 нс (то есть частота работы $10$0 MHz), максимальная тактовая частота стала следующей: Fmax = $489.24$ MHz, Restricted Fmax = $250$ MHz. Из результатов видно, что значение Fmax увеличилось.

В таблице \textbf{Multicorner Timing Analysis Summary} значения задержек стали равны:
\begin{itemize}
\setlength\itemsep{0em}
\item Setup Slack = $-1.044$ нс.
\item Hold Slack = $0.256$ нс.
\item Slack для Minimum Pulse Width = $-3.000$ нс.
\end{itemize}
%TODO Объясните с какими параметрами возникли проблемы

\subsubsection{Анализ временных диаграмм}

На рис. \ref{fig:waveform} изображена временная диаграмма распространения данных и тактового сигнала.

\begin{figure}[H]
\begin{center}
	\includegraphics[scale=0.9]{waveform_setup}
	\caption{Временная диаграмма Setup анализа}
	\label{fig:waveform}
\end{center}
\end{figure}
%TODO Найдите на ней параметры, использованные в лекции для анализа зазора Setup Slack.

\begin{figure}[H]
\begin{center}
	\includegraphics[scale=0.9]{waveform_hold}
	\caption{Временная диаграмма Hold анализа}
	\label{fig:waveform}
\end{center}
\end{figure}
%TODO Найдите на ней параметры, использованные в лекции для анализа зазоров Slack Hold

\section{Выводы}

\end{document}