\documentclass[a4paper,14pt]{extarticle}

\usepackage[utf8x]{inputenc}
\usepackage[T1,T2A]{fontenc}
\usepackage[russian]{babel}
\usepackage{hyperref}
\usepackage{indentfirst}
\usepackage{here}
\usepackage{array}
\usepackage{graphicx}
\usepackage{caption}
\usepackage{subcaption}
\usepackage{chngcntr}
\usepackage{amsmath}
\usepackage{amssymb}
\usepackage{pgfplots}
\usepackage{pgfplotstable}
\usepackage[left=2cm,right=2cm,top=2cm,bottom=2cm,bindingoffset=0cm]{geometry}
\usepackage{multicol}
\usepackage{askmaps}
\usepackage{titlesec}

\renewcommand{\le}{\ensuremath{\leqslant}}
\renewcommand{\leq}{\ensuremath{\leqslant}}
\renewcommand{\ge}{\ensuremath{\geqslant}}
\renewcommand{\geq}{\ensuremath{\geqslant}}
\renewcommand{\epsilon}{\ensuremath{\varepsilon}}
\renewcommand{\phi}{\ensuremath{\varphi}}
\renewcommand{\thefigure}{\arabic{figure}} 	
\renewcommand*\not[1]{\overline{#1}}

\titleformat*{\section}{\large\bfseries} 
\titleformat*{\subsection}{\normalsize\bfseries} 
\titleformat*{\subsubsection}{\normalsize\bfseries} 
\titleformat*{\paragraph}{\normalsize\bfseries} 
\titleformat*{\subparagraph}{\normalsize\bfseries} 

\counterwithin{figure}{section}
\counterwithin{equation}{section}
\counterwithin{table}{section}
\newcommand{\sign}[1][5cm]{\makebox[#1]{\hrulefill}}
\graphicspath{{../pics/}}
\captionsetup{justification=centering,margin=1cm}
\def\arraystretch{1.3}
\setlength\parindent{5ex}
\titlelabel{\thetitle.\quad}

\begin{document}

\begin{titlepage}
\begin{center}
	Санкт-Петербургский Политехнический Университет Петра Великого\\[3mm]
	Институт компьютерных наук и технологий \\[3mm]
	Кафедра компьютерных систем и программных технологий\\[6cm]
	
	\textbf{КУРСОВОЙ ПРОЕКТ}\\[3mm]
	\textbf{Разработка устройства передачи данных}\\[3mm]
	по дисциплине <<Автоматизация проектирования\\[3mm] 
	дискретных устройств>>\\[6cm]
\end{center}

\begin{flushleft}
	Выполнил\\[3mm]
	студент гр. 33501/4  \hspace*{7.1cm} Дьячков В.В.\\[3mm]
	Руководитель\\[3mm]
	доцент, к.т.н. \hspace*{8.5cm} А.С. Филиппов\\[4mm]
	\hspace*{11.5cm} <<\sign[7mm]>> \sign[20mm] \the\year\hspace{1mm} г.
\end{flushleft}

\vfill

\begin{center}
	Санкт-Петербург\\
	\the\year
\end{center}
\end{titlepage}

\addtocounter{page}{1}
\counterwithin{lstlisting}{section}

\tableofcontents
\listoffigures
\lstlistoflistings
\newpage

\section{Задание}

На языке SystemVerilog создайте структурное описание параметризированного устройства (параметр \code{INV = 1} -- инверсия выходных данных), приведенного на рис. \ref{fig:elab7_1_0}.

\begin{figure}[H]
\begin{center}
	\includegraphics[width=0.71\textwidth]{elab7_1_0}
	\caption{Параметризированное устройство}
	\label{fig:elab7_1_0}
\end{center}
\end{figure}
\vspace{-0.5cm}

В состав устройства входят:
\begin{itemize}
	\item \code{cnt_d} -- cчетчик делитель (параметр -- \code{DIV}), обеспечивающий счет по модулю \code{DIV} (базовое значение -- $3$) и формирование синхронного сигнала переноса (активный уровень сигнала -- $1$, длительность один такт тактовой частоты) по достижению счетчиком значения \code{DIV - 1}.
	\item \code{data} -- формирователь данных для модуля памяти (реализован на базе параметризированного счетчика \code{cnt_N}).
	\item \code{adr} -- формирователь адреса для модуля памяти (реализован на базе параметризированного счетчика \code{cnt_N}).
	\item \code{RAM} -- модуль памяти (параметры: \code{word_num} -- число слов (базовое значение $16$), \code{data_W} -- разрядность данных (базовое значение $4$); простая одно портовая память с чтением новых данных в процессе записи):
		\begin{itemize}
			\item Для расчета разрядности шины адреса следует использовать функцию с постоянным значением для вычисления $\log_2($\code{word_num}$)$;
			\item Вход \code{wre} – вход разрешения записи ($=1$ -- запись в память разрешена);
			\item Модуль памяти должен быть инициализирован данными: \code{4'h0} -- четные адреса; \code{4'hf} -- нечетные адреса.
		\end{itemize}
	\item \code{cnt_N} -- двоичный счетчик на сложение с параметризированной разрядностью (параметр \code{N}, базовое значение -- $4$), имеющий вход тактовых сигналов (\code{clk}), вход разрешения работы (\code{ena}), вход асинхронного сброса (\code{rst}) и выход -- \code{q[N-1:0]}.
	\item Входы:
		\begin{itemize}
			\item Кнопка \code{pbb} -- вход асинхронного сброса (кнопка нажата – сброс);
			\item Кнопка \code{pba} -- вход разрешения записи в память (кнопка нажата – запись разрешена);
			\item Тактовый сигнал (\code{clk}) подается от тактового генератора. Частота тактового сигнала -- 25МГц.
		\end{itemize}
	\item Выходы: светодиоды \code{led[3:0]} -- выходы устройства.
\end{itemize}

\section{Код на языке SystemVerilog}

В листингах \ref{code:4} -- \ref{code:7} приведен код программы на языке SystemVerilog.

\lstinputlisting[caption=elab7\_1.sv, label=code:4]{elab7_1.sv}

\lstinputlisting[caption=cnt\_d.sv, label=code:5]{cnt_d.sv}

\lstinputlisting[caption=cnt\_N.sv, label=code:6]{cnt_N.sv}

\lstinputlisting[caption=RAM.sv, label=code:7]{RAM.sv}

\section{Результаты синтеза}

На рис. \ref{fig:elab7_1_rtl} -- \ref{fig:elab7_1_rtl_3} приведены изображения синтезированного устройства и каждого модуля по отдельности в RLT Viewer.

\begin{figure}[H]
\begin{center}
	\includegraphics[width=\textwidth]{elab7_1_rtl}
	\caption{Результат синтеза устройства в RLT Viewer}
	\label{fig:elab7_1_rtl}
\end{center}
\end{figure}

\begin{figure}[H]
\begin{center}
	\includegraphics[width=1\textwidth]{elab7_1_rtl_1}
	\caption{Результат синтеза \code{cnt_div} в RLT Viewer}
	\label{fig:elab7_1_rtl_1}
\end{center}
\end{figure}

\begin{figure}[H]
\begin{center}
	\includegraphics[width=0.7\textwidth]{elab7_1_rtl_2}
	\caption{Результат синтеза \code{cnt_addr} в RLT Viewer}
	\label{fig:elab7_1_rtl_2}
\end{center}
\end{figure}

\begin{figure}[H]
\begin{center}
	\includegraphics[width=0.7\textwidth]{elab7_1_rtl_3}
	\caption{Результат синтеза \code{cnt_data} в RLT Viewer}
	\label{fig:elab7_1_rtl_3}
\end{center}
\end{figure}

\section{Результаты моделирования}
\label{sec:elab7_1_modeling}

На рис. \ref{fig:elab7_1_modeling_1} -- \ref{fig:elab7_1_modeling} изображены временные диаграммы работы каждого модуля по отдельности и всего синтезированного устройства при \code{DIV = 3},\\ \code{N = 4}, \code{word_num = 16}, \code{data_W = 4}, \code{INV = 0}.

\begin{figure}[H]
\begin{center}
	\includegraphics[width=\textwidth]{elab7_1_modeling_1}
	\caption{Результаты моделирования модуля \code{cnt_d}}
	\label{fig:elab7_1_modeling_1}
\end{center}
\end{figure}

\begin{figure}[H]
\begin{center}
	\includegraphics[width=\textwidth]{elab7_1_modeling_2}
	\caption{Результаты моделирования модуля \code{cnt_N}}
	\label{fig:elab7_1_modeling_2}
\end{center}
\end{figure}

\begin{figure}[H]
\begin{center}
	\includegraphics[width=\textwidth]{elab7_1_modeling_3}
	\caption{Результаты моделирования модуля \code{RAM}}
	\label{fig:elab7_1_modeling_3}
\end{center}
\end{figure}

\begin{figure}[H]
\begin{center}
	\includegraphics[width=\textwidth]{elab7_1_modeling}
	\caption{Результаты моделирования устройства}
	\label{fig:elab7_1_modeling}
\end{center}
\end{figure}

При моделировании устройства при активном уровне \code{pba} происходит запись в память, а при активном уровне \code{pbb} происходит асинхронный сброс. Сначала происходит чтение начальных данных, затем запись в память, после чего происходит асинхронный сброс и чтение записанных ранее данных.

\section{Назначение выводов СБИС}

На рис. \ref{fig:elab7_1_pins} приведены назначения выводов СБИС в Pin Planner.

\begin{figure}[H]
\begin{center}
	\includegraphics{elab7_1_pins}
	\caption{Таблица назначений в Pin Planer}
	\label{fig:elab7_1_pins}
\end{center}
\end{figure}

\section{Результаты проверки на плате}

Для тестирования проекта на плате были использованы тесты, описанные в пункте \ref{sec:elab7_1_modeling} при параметрах \code{DIV = 25_000_000}, \code{N = 4}, \code{word_num = 16}, \code{data_W = 4}, \code{INV = 1}. Результаты тестирования совпадают с ожидаемыми, следовательно, устройство работает верно.

\section{Выводы}

Реализовано описание параметризированного устройства, включающего счетчик-делитель, формирователи данных и адреса, модуль памяти и двоичный счетчик на сложение. Результаты моделирования и тестирования на плате показали, что разработанное устройство работает верно.

\end{document}