\documentclass[a4paper,14pt]{extarticle}

\usepackage[utf8x]{inputenc}
\usepackage[T1,T2A]{fontenc}
\usepackage[russian]{babel}
\usepackage{hyperref}
\usepackage{indentfirst}
\usepackage{here}
\usepackage{array}
\usepackage{graphicx}
\usepackage{caption}
\usepackage{subcaption}
\usepackage{chngcntr}
\usepackage{amsmath}
\usepackage{amssymb}
\usepackage{pgfplots}
\usepackage{pgfplotstable}
\usepackage[left=2cm,right=2cm,top=2cm,bottom=2cm,bindingoffset=0cm]{geometry}
\usepackage{multicol}
\usepackage{askmaps}
\usepackage{titlesec}

\renewcommand{\le}{\ensuremath{\leqslant}}
\renewcommand{\leq}{\ensuremath{\leqslant}}
\renewcommand{\ge}{\ensuremath{\geqslant}}
\renewcommand{\geq}{\ensuremath{\geqslant}}
\renewcommand{\epsilon}{\ensuremath{\varepsilon}}
\renewcommand{\phi}{\ensuremath{\varphi}}
\renewcommand{\thefigure}{\arabic{figure}} 	
\renewcommand*\not[1]{\overline{#1}}

\titleformat*{\section}{\large\bfseries} 
\titleformat*{\subsection}{\normalsize\bfseries} 
\titleformat*{\subsubsection}{\normalsize\bfseries} 
\titleformat*{\paragraph}{\normalsize\bfseries} 
\titleformat*{\subparagraph}{\normalsize\bfseries} 

\counterwithin{figure}{section}
\counterwithin{equation}{section}
\counterwithin{table}{section}
\newcommand{\sign}[1][5cm]{\makebox[#1]{\hrulefill}}
\graphicspath{{../pics/}}
\captionsetup{justification=centering,margin=1cm}
\def\arraystretch{1.3}
\setlength\parindent{5ex}
\titlelabel{\thetitle.\quad}

\begin{document}

\begin{titlepage}
\begin{center}
	Санкт-Петербургский Политехнический Университет Петра Великого\\[3mm]
	Институт компьютерных наук и технологий \\[3mm]
	Кафедра компьютерных систем и программных технологий\\[6cm]
	
	\textbf{КУРСОВОЙ ПРОЕКТ}\\[3mm]
	\textbf{Разработка устройства передачи данных}\\[3mm]
	по дисциплине <<Автоматизация проектирования\\[3mm] 
	дискретных устройств>>\\[6cm]
\end{center}

\begin{flushleft}
	Выполнил\\[3mm]
	студент гр. 33501/4  \hspace*{7.1cm} Дьячков В.В.\\[3mm]
	Руководитель\\[3mm]
	доцент, к.т.н. \hspace*{8.5cm} А.С. Филиппов\\[4mm]
	\hspace*{11.5cm} <<\sign[7mm]>> \sign[20mm] \the\year\hspace{1mm} г.
\end{flushleft}

\vfill

\begin{center}
	Санкт-Петербург\\
	\the\year
\end{center}
\end{titlepage}

\addtocounter{page}{1}
\counterwithin{lstlisting}{section}

\tableofcontents
\lstlistoflistings
\listoffigures
\newpage

\section{Задание}

\begin{itemize}
	\item Для задания \code{lab5_2} осеннего семестра на языке Verilog создать описание тестов:
		\begin{itemize}
			\item Тест класса 1 -- \code{tb1_labs_12.v} (значения входов, выходов и время записывать в файл используя любую команду, данные должны быть отформатированы для удобного чтения, имя файла -- \code{tb1_.dat});
			\item Тест класса 2 с вычислением результата -- \code{tb2_labs_12.v};
			\item Тест класса 2 с чтением файлов -- \code{tb2f_labs_12.v} (файл с тестовыми воздействиями -- \code{input_labs_12.dat}, файл с ожидаемыми результатами -- \code{exp_labs_12.dat}.
		\end{itemize}

	\item Осуществить проверку модуля с использованием всех тестов; проверку тестов класса 2 на обработку ошибок.
\end{itemize}

\section{Описание устройства}

В листинге \ref{code:lab5_2} приведено описание тестируемого устройства, включающего счетчик-делитель, обеспечивающий счет по модулю $N$ и формирующий сигнал переноса, и конечный автомат.
\lstinputlisting[caption=\code{lab5_2.v}, label=code:lab5_2]{lab5_2.v}

\section{Описание тестов}
\label{sec:tests}

\subsection{Тест первого класса}

В листинге \ref{code:test1} приведено описание теста первого класса.
\lstinputlisting[caption=\code{tb1_labs_12.v}, label=code:test1]{tb1_labs_12.v}

В листинге \ref{code:test1_results} приведен вывод результатов теста в файл.
\lstinputlisting[caption=\code{tb1_.dat}, label=code:test1_results, style=console, linerange={1-5, 26-301}]{tb1_.dat}

На рис. \ref{fig:test1_results} изображена временная диаграмма теста.
\begin{figure}[H]
	\begin{center}
		\includegraphics[width=0.95\textwidth]{test1_results}
		\caption{Результаты теста первого класса}
		\label{fig:test1_results}
	\end{center}
\end{figure}

\subsection{Тест второго класса с вычислением результата}

В листинге \ref{code:test2} приведено описание теста второго класса с вычислением результата.
\lstinputlisting[caption=\code{tb2_labs_12.v}, label=code:test2]{tb2_labs_12.v}

На рис. \ref{fig:test2_results} изображена временная диаграмма теста.
\begin{figure}[H]
	\begin{center}
		\includegraphics[width=0.95\textwidth]{test2_results}
		\caption{Результаты теста второго класса с вычислением результата}
		\label{fig:test2_results}
	\end{center}
\end{figure}

В листинге \ref{code:test2_error} приведен вывод результатов теста в консоль при внесении ошибки в вычисление ожидаемого значения.	
\begin{lstlisting}[caption=Результаты ошибочного теста второго класса с вычислением результата, label=code:test2_error, style=console]
run -all
# Expected 2, got 1
# Expected 3, got 2
# Expected 4, got 3
# Expected 5, got 4
# Expected 6, got 5
...
# Expected 4, got 2
# Expected 5, got 3
# Expected 6, got 4
# Expected 7, got 5
# Expected 0, got 6
#    Time: 630 ns  Iteration: 0  Instance: /tb2_labs_12
\end{lstlisting}

\subsection{Тест второго класса с чтением файлов}

В листинге \ref{code:test3} приведено описание теста второго класса с чтением файлов. В данном тесте на вход устройства из файла \code{input_labs_12.dat}, содержание которого приведено в листинге \ref{code:input}, подаются всевозможные значения, а ожидаемые считываются из файла \code{exp_labs_12.dat}, содержание которого приведено в листинге \ref{code:exp}.

\newpage

\lstinputlisting[caption=\code{tb2f_labs_12.v}, label=code:test3]{tb2f_labs_12.v}
\begin{multicols}{2}
	\lstinputlisting[caption=\code{inp_labs_12.dat}, linerange={1-8}, label=code:input, style=dat]{input_labs_12.dat}	
	\lstinputlisting[caption=\code{exp_labs_12.dat}, linerange={1-8}, label=code:exp, style=dat]{exp_labs_12.dat}
\end{multicols}

\newpage

На рис. \ref{fig:test3_results} изображена временная диаграмма теста.
\begin{figure}[H]
	\begin{center}
		\includegraphics[width=0.85\textwidth]{test3_results}
		\caption{Результаты теста второго класса с чтением файлов}
		\label{fig:test3_results}
	\end{center}
\end{figure}

В листинге \ref{code:test3_error} приведен вывод результатов теста в консоль при внесении ошибок в ожидаемые значения.
\begin{lstlisting}[caption=Результаты ошибочного теста второго класса с чтением файлов, label=code:test3_error, style=console]
run -all
# Expected 0, got 1
# Expected 1, got 2
# Expected 2, got 1
# Expected 1, got 0
# Expected 0, got 2
# Expected 2, got 4
#    Time: 170 ns  Iteration: 1  Instance: /tb2f_labs_12
\end{lstlisting}

\section{Выводы}

В ходе лабораторной работы на языке Verilog описаны тесты первого и второго классов для устройства, включающего счетчик-делитель и конечный автомат. Тестирование разработанного устройства показало, что результаты совпадают с ожидаемыми и устройство работает верно.

\end{document}