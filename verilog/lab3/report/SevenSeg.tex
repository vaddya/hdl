\documentclass[12pt]{article}
\usepackage[T1]{fontenc} 
\usepackage{tikz,xifthen}

\usepackage{SevenSeg}

\usepackage[a4paper,body={19cm,24cm},
                    head=1cm,
                    footskip=2cm]{geometry}
                    
\title{7 segment displayer with PGF/TIKZ}                     
\author{Germain Gondor}                                  

\date{\today}

\makeindex

\begin{document}

\maketitle

\section{Outlook}

This package has been built to draw seven segment displayer with PGF/TIKZ. Segments are directly displayed in the right color to show the  hexad�cimal number $(0,\ldots,9,A,\ldots,F)$ or decimal $(0,\ldots,15)$.\\

Unzip the package to your local folder.\\

Put on top of your file \verb"\usepackage{SevenSeg}" just below \verb"\documentclass{\ldots}".

\section{Commandes}

\subsection{Segment letters}
\vspace{1em}
\begin{minipage}{0.47\textwidth}
To name each segment of the displayer, use the command \verb"\SSGLeg[size]{position}":
\begin{itemize}
	\item \verb"size". This defines the size. Default is 3em
	\item \verb"position" refers to an existing point. To select the point $(0,0)$ just write $\left\{\right\}$\end{itemize}
\end{minipage}\hfill\begin{minipage}{0.25\textwidth}
\begin{tikzpicture}
\coordinate(A) at(5em,0);
\SSGLeg{}
\SSGLeg[2em]{A}
\end{tikzpicture}\\
\end{minipage}\begin{minipage}{0.25\textwidth}
\begin{verbatim}
\begin{tikzpicture}
\coordinate(A) at(5em,0);
\SSGLeg{}
\SSGLeg[2em]{A}
\end{tikzpicture}
\end{verbatim}
\end{minipage}
	
\subsection{Seven segment displayer}

\begin{minipage}{0.7\textwidth}
The command \verb"\SSGNb[size]{position}{number}" display the cell with the right lighted segments. Arguments allow to change:
\begin{itemize}
	\item the \verb"size" . PDefault is 3em
	\item the \verb"position". Position refers to an existing point. To select the point $(0,0)$, just write $\left\{\right\}$
	\item the \verb"number". Number should be in $0$ and $15$ or $A$ et $F$
\end{itemize}

\end{minipage}\hfill\begin{minipage}{0.28\textwidth}
\begin{tikzpicture}
\SSGNb[1cm]{}{B}
\coordinate(A)at(5em,0);
\SSGNb[2em]{A}{11}
\end{tikzpicture}\\
%\end{minipage}\hfill\begin{minipage}{0.28\textwidth}
\begin{verbatim}
\begin{tikzpicture}
\SSGNb[2cm]{}{B}
\coordinate(A)at(5em,0);
\SSGNb[2em]{A}{11}
\end{tikzpicture}
\end{verbatim}
\end{minipage}\\


\subsection{Logical cell}

To put a box arround the cell, just use the command  \verb"SSGBox[Style]{position}" with 
\begin{itemize}
	\item \verb"Style" to change the style. Default is \verb"line width=2pt"
	\item  \verb"position" refers to an existing point. To select the point $(0,0)$, just write $\left\{\right\}$\end{itemize}

To change the cell into a seven segment display with binary connections, use  \verb"SSGDCB[Style]{position}".\\
To connect the cell,  \verb"positionBit0", \verb"positionBit1", \verb"positionBit2" and  \verb"positionBit3" refers to the connecting points.\\
 
\begin{minipage}{0.28\textwidth}
\begin{tikzpicture}
\coordinate(A)at(6em,1cm);
\SSGNb{A}{F}
\SSGBox{A}
\def\taille{1.5cm}
\SSGNb[\taille]{}{5}
\SSGDCB[line width=2pt, blue]{}
\foreach \x in {0,...,3}{\node at(Bit\x)[below]{$B_\x$};}
\end{tikzpicture}
\end{minipage}\hfill\begin{minipage}{0.38\textwidth}
\begin{verbatim}
\begin{tikzpicture}
\coordinate(A)at(6em,1cm);
\SSGNb{A}{F}
\SSGBox{A}
\def\taille{1.5cm}
\SSGNb[\taille]{}{5}
\SSGDCB[line width=2pt, blue]{}
\foreach \x in {0,...,3}
{\node at(Bit\x)[below]{$B_\x$};}
\end{tikzpicture}\end{verbatim}
\end{minipage}\\

\section{Options}

\subsection{Slanted cell}

\begin{minipage}{0.4\textwidth}
To slant the cell, use the  \verb"xslant" option of PGF/TIKZ, or use a \verb"scope" environement  \verb"\tikzpicture":
\end{minipage}\hfill\begin{minipage}{0.18\textwidth}
\begin{tikzpicture}[xslant=0.1]
\SSGNb[2em]{}{5}
\SSGDCB{}
\end{tikzpicture}
\end{minipage}\hfill\begin{minipage}{0.38\textwidth}
\begin{verbatim}
\begin{tikzpicture}[xslant=0.1]
\SSGNb{}{5}
\SSGDCB{}
\end{tikzpicture}
\end{verbatim}
\end{minipage}\\

\subsection{Styles}

The style is given by \textit{SSGSty}. The style of on (off) segments is \textit{SSGOn} (resp. \textit{SSGOff}).\\

\begin{verbatim}
\tikzstyle SSGSty=[line cap=round]
\tikzstyle SSGOn=[green,line width=3pt]
\tikzstyle SSGOff=[gray!20!white,line width=3pt]
\end{verbatim}

\section{Examples}

\begin{minipage}{0.48\textwidth}
\begin{tikzpicture}[scale=0.75]
\foreach \x in{0,...,3}
{\coordinate (L\x) at({(\x-0)*2.5} ,-3*1);
\SSGNb{L\x}{\x}}
\foreach \x in{4,...,7}
{\coordinate (L\x) at({(\x-4)*2.5} ,-3*2);
\SSGNb{L\x}{\x}}
\begin{scope}[xslant=0.1]
\foreach \x in{8,...,11}
{\coordinate(L\x) at({(\x-8)*2.5} ,-3*3);
\SSGNb{L\x}{\x}}
\end{scope}
\foreach \x in{12,...,15}
{\coordinate (L\x) at({(\x-12)*2.5} ,-3*4);
\SSGNb{L\x}{\x}}
\end{tikzpicture}\end{minipage}\hfill\begin{minipage}{0.48\textwidth}
\begin{verbatim}
\begin{tikzpicture}[scale=0.8]
\foreach \x in{0,...,3}
{\coordinate (L\x) at({(\x-0)*2.5} ,-3*1);
\SSGNb{L\x}{\x}}
\foreach \x in{4,...,7}
{\coordinate (L\x) at({(\x-4)*2.5} ,-3*2);
\SSGNb{L\x}{\x}}
\begin{scope}[xslant=0.1]
\foreach \x in{8,...,11}
{\coordinate(L\x) at({(\x-8)*2.5} ,-3*3);
\SSGNb{L\x}{\x}}
\end{scope}
\foreach \x in{12,...,15}
{\coordinate (L\x) at({(\x-12)*2.5} ,-3*4);
\SSGNb{L\x}{\x}}
\end{tikzpicture}
\end{verbatim}
\end{minipage}\\


 \begin{verbatim}
\begin{tikzpicture}
\coordinate(A)at(-6em,0);
\SSGLeg{A}
\begin{scope}[xslant=0.1]
\SSGNb{}{3}
\SSGDCB{}
\end{scope}
\foreach \x in{0,...,1}
{\fill[purple](-7em,-7em-\x*1em)circle(3pt)coordinate(B\x);
\draw[purple,very thick](B\x)node[left]{$B_\x$}--++(14em,0)-|(Bit\x);
\fill[purple](Bit\x|-B\x)circle(3pt);}
\foreach \x in{2,...,3}
{\fill(-7em,-7em-\x*1em)circle(3pt)coordinate(B\x);
\draw[very thick](B\x)node[left]{$B_\x$}--++(14em,0)-|(Bit\x);
\fill(Bit\x|-B\x)circle(3pt);}
\end{tikzpicture}
\end{verbatim}


\begin{tikzpicture}[scale=0.8]
\coordinate(A)at(-6em,0);
\SSGLeg{A}
\begin{scope}[xslant=0.1]
\SSGNb{}{3}
\SSGDCB{}
\end{scope}
\foreach \x in{0,...,1}
{\fill[purple](-7em,-7em-\x*1em)circle(3pt)coordinate(B\x);
\draw[purple,very thick](B\x)node[left]{$B_\x$}--++(14em,0)-|(Bit\x);
\fill[purple](Bit\x|-B\x)circle(3pt);}
\foreach \x in{2,...,3}
{\fill(-7em,-7em-\x*1em)circle(3pt)coordinate(B\x);
\draw[very thick](B\x)node[left]{$B_\x$}--++(14em,0)-|(Bit\x);
\fill(Bit\x|-B\x)circle(3pt);}
\end{tikzpicture}



\end{document}