\documentclass[a4paper,14pt]{extarticle}

\usepackage[utf8x]{inputenc}
\usepackage[T1,T2A]{fontenc}
\usepackage[russian]{babel}
\usepackage{hyperref}
\usepackage{indentfirst}
\usepackage{here}
\usepackage{array}
\usepackage{graphicx}
\usepackage{caption}
\usepackage{subcaption}
\usepackage{chngcntr}
\usepackage{amsmath}
\usepackage{amssymb}
\usepackage{pgfplots}
\usepackage{pgfplotstable}
\usepackage[left=2cm,right=2cm,top=2cm,bottom=2cm,bindingoffset=0cm]{geometry}
\usepackage{multicol}
\usepackage{askmaps}
\usepackage{titlesec}

\renewcommand{\le}{\ensuremath{\leqslant}}
\renewcommand{\leq}{\ensuremath{\leqslant}}
\renewcommand{\ge}{\ensuremath{\geqslant}}
\renewcommand{\geq}{\ensuremath{\geqslant}}
\renewcommand{\epsilon}{\ensuremath{\varepsilon}}
\renewcommand{\phi}{\ensuremath{\varphi}}
\renewcommand{\thefigure}{\arabic{figure}} 	
\renewcommand*\not[1]{\overline{#1}}

\titleformat*{\section}{\large\bfseries} 
\titleformat*{\subsection}{\normalsize\bfseries} 
\titleformat*{\subsubsection}{\normalsize\bfseries} 
\titleformat*{\paragraph}{\normalsize\bfseries} 
\titleformat*{\subparagraph}{\normalsize\bfseries} 

\counterwithin{figure}{section}
\counterwithin{equation}{section}
\counterwithin{table}{section}
\newcommand{\sign}[1][5cm]{\makebox[#1]{\hrulefill}}
\graphicspath{{../pics/}}
\captionsetup{justification=centering,margin=1cm}
\def\arraystretch{1.3}
\setlength\parindent{5ex}
\titlelabel{\thetitle.\quad}

\begin{document}

\begin{titlepage}
\begin{center}
	Санкт-Петербургский Политехнический Университет Петра Великого\\[3mm]
	Институт компьютерных наук и технологий \\[3mm]
	Кафедра компьютерных систем и программных технологий\\[6cm]
	
	\textbf{КУРСОВОЙ ПРОЕКТ}\\[3mm]
	\textbf{Разработка устройства передачи данных}\\[3mm]
	по дисциплине <<Автоматизация проектирования\\[3mm] 
	дискретных устройств>>\\[6cm]
\end{center}

\begin{flushleft}
	Выполнил\\[3mm]
	студент гр. 33501/4  \hspace*{7.1cm} Дьячков В.В.\\[3mm]
	Руководитель\\[3mm]
	доцент, к.т.н. \hspace*{8.5cm} А.С. Филиппов\\[4mm]
	\hspace*{11.5cm} <<\sign[7mm]>> \sign[20mm] \the\year\hspace{1mm} г.
\end{flushleft}

\vfill

\begin{center}
	Санкт-Петербург\\
	\the\year
\end{center}
\end{titlepage}

\addtocounter{page}{1}
\counterwithin{lstlisting}{section}

\tableofcontents
%\listoffigures
\newpage

\section{Цель работы}

\begin{itemize}
	\item Получение навыков анализа простейших процессов с использованием базовых функций встраиваемого логического анализатора SignalTap II и ISSPE-мегафункции приема и установки сигналов через JTAG в пакете Quartus II;
	\item Исследование работы варианта реализации широтно-импульсного модулятора с использованием средств системной отладки пакета Quartus II.
\end{itemize}

\section{Ход работы}

\subsection{Создание и ввод проекта}

На рис. \ref{fig:source} изображена схема исследуемого объекта -- широтно-импульсного модулятора (счетчик \code{Counter_PWM}, компаратор \code{Compare_PWM} и \code{DFF}-триггер). Счетчик работает с тактовой частотой 25МГц. Сигнал с выхода триггера подается на вывод, подключенный к \code{LED}. Рабочие данные поступают на ШИМ с входных выводов \code{D[]}.

Для отладки ШИМ в состав проекта введены мультиплексор данных \code{Mux_Data} и блок \code{ISSPE}, позволяющий в процессе отладки исключить системное окружение, задающее сигналы на выводы \code{D[]}. Данные на компаратор поступают с выхода мультиплексора \code{Mux_Data}, который в соответствии с управляющим сигналом \code{Sel} коммутирует на свой выход либо внешние данные \code{D[]} (сигналы на \code{D[]} подаются с переключателей \code{SW[7..0]}), либо внутреннюю шину данных \code{DI[]}. При отладке ШИМ на лабораторном стенде создание управляющего сигнала \code{Sel} и формирование кода на внутренней шине данных \code{DI[]} обеспечивается с помощью мегафункции In-System Sources and Probe Editor (ISSPE). ISSPE также подключен для наблюдения сигналов на входах компаратора (\code{Q[]}, \code{D_Mux[]}).

\begin{figure}[H]
	\begin{center}
		\includegraphics[width=\textwidth]{source}
		\caption{Исследуемый проект}
		\label{fig:source}
	\end{center}
\end{figure}
\vspace{-1cm}

\section{Выводы}



\end{document}