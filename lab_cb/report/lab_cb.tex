\documentclass[a4paper,14pt]{extarticle}

\usepackage[utf8x]{inputenc}
\usepackage[T1,T2A]{fontenc}
\usepackage[russian]{babel}
\usepackage{hyperref}
\usepackage{indentfirst}
\usepackage{here}
\usepackage{array}
\usepackage{graphicx}
\usepackage{caption}
\usepackage{subcaption}
\usepackage{chngcntr}
\usepackage{amsmath}
\usepackage{amssymb}
\usepackage{pgfplots}
\usepackage{pgfplotstable}
\usepackage[left=2cm,right=2cm,top=2cm,bottom=2cm,bindingoffset=0cm]{geometry}
\usepackage{multicol}
\usepackage{askmaps}
\usepackage{titlesec}

\renewcommand{\le}{\ensuremath{\leqslant}}
\renewcommand{\leq}{\ensuremath{\leqslant}}
\renewcommand{\ge}{\ensuremath{\geqslant}}
\renewcommand{\geq}{\ensuremath{\geqslant}}
\renewcommand{\epsilon}{\ensuremath{\varepsilon}}
\renewcommand{\phi}{\ensuremath{\varphi}}
\renewcommand{\thefigure}{\arabic{figure}} 	
\renewcommand*\not[1]{\overline{#1}}

\titleformat*{\section}{\large\bfseries} 
\titleformat*{\subsection}{\normalsize\bfseries} 
\titleformat*{\subsubsection}{\normalsize\bfseries} 
\titleformat*{\paragraph}{\normalsize\bfseries} 
\titleformat*{\subparagraph}{\normalsize\bfseries} 

\counterwithin{figure}{section}
\counterwithin{equation}{section}
\counterwithin{table}{section}
\newcommand{\sign}[1][5cm]{\makebox[#1]{\hrulefill}}
\graphicspath{{../pics/}}
\captionsetup{justification=centering,margin=1cm}
\def\arraystretch{1.3}
\setlength\parindent{5ex}
\titlelabel{\thetitle.\quad}

\begin{document}

\begin{titlepage}
\begin{center}
	Санкт-Петербургский Политехнический Университет Петра Великого\\[3mm]
	Институт компьютерных наук и технологий \\[3mm]
	Кафедра компьютерных систем и программных технологий\\[6cm]
	
	\textbf{КУРСОВОЙ ПРОЕКТ}\\[3mm]
	\textbf{Разработка устройства передачи данных}\\[3mm]
	по дисциплине <<Автоматизация проектирования\\[3mm] 
	дискретных устройств>>\\[6cm]
\end{center}

\begin{flushleft}
	Выполнил\\[3mm]
	студент гр. 33501/4  \hspace*{7.1cm} Дьячков В.В.\\[3mm]
	Руководитель\\[3mm]
	доцент, к.т.н. \hspace*{8.5cm} А.С. Филиппов\\[4mm]
	\hspace*{11.5cm} <<\sign[7mm]>> \sign[20mm] \the\year\hspace{1mm} г.
\end{flushleft}

\vfill

\begin{center}
	Санкт-Петербург\\
	\the\year
\end{center}
\end{titlepage}

\addtocounter{page}{1}
\counterwithin{lstlisting}{section}

\tableofcontents
\listoffigures
\newpage

\section{Цель работы}

\begin{itemize}
	\item Закрепление знаний, полученных в SignalTapII 1, продолжение изучения возможностей встроенного логического анализатора SignalTapII пакета QuartusII;
	\item Получение навыков анализа реальных процессов с помощью логического анализатора SignalTapII пакета QuartusII;
	\item Исследование дребезга контактов движкового переключателя с помощью логического анализатора;
	\item Реализация и исследование работы схемы борьбы с дребезгом контактов.
\end{itemize}

\section{Подготовка проекта для исследования дребезга контактов движкового переключателя}

\subsection{Создание и ввод проекта}

На рис. \ref{fig:source} изображена схема для исследования дребезга контактов механического движкового переключателя, дополненная логическим анализатором, фиксирующим требуемые сигналы в схеме. Движковый переключатель \code{sw} подключен к синхровходу восьмиразрядного счетчика \code{cnt}. Счетчик подсчитывает число переходов \code{0=>1} при переключении движкового переключателя из \code{0} в \code{1} и обратно. Число переходов (импульсов дребезга) отображается на светодиодах \code{led[7..0]}. Сигнал от кнопки \code{pb} осуществляет сброс счетчика.

\begin{figure}[H]
	\begin{center}
		\includegraphics[scale=0.95]{source}
		\caption{Исследуемый проект}
		\label{fig:source}
	\end{center}
\end{figure}
\vspace{-1cm}

\subsection{Создание экземпляра логического анализатора ELA1}

На рис. \ref{fig:stp_1} изображено окно SignalTapII Logic Analyzer со списком целей в созданном \code{stp}-файле.
\vspace{-0.5cm}
\begin{figure}[H]
	\begin{center}
		\includegraphics[width=0.9\textwidth]{stp_1}
		\caption{Список целей в SignalTapII Logic Analyzer}
		\label{fig:stp_1}
	\end{center}
\end{figure}

\subsection{Конфигурирование СБИС}

На рис. \ref{fig:stp_2} изображено окно Instance Manager со статусом \code{Ready to acquire}.
\vspace{-0.5cm}
\begin{figure}[H]
	\begin{center}
		\includegraphics[width=1\textwidth]{stp_2}
		\caption{Instance Manager}
		\label{fig:stp_2}
	\end{center}
\end{figure}

\section{Исследование дребезга контактов движкового переключателя}
\label{sec:cb}

\subsection{Исследование дребезга при переключениях из 0 => 1}

Переведем \code{sw[0]} в положение 0 и сбросим счетчик с помощью кнопки \code{pbb}. Проведем эксперимент, переключая \code{sw[0]} из 0 в 1, будем подсчитывать количество импульсов $N_{stp}$ на плате, отображаемых на светодиодах, количество импульсов $N_{cnt}$, подсчитанных с помощью логического анализатора ELA1 и длительность процесса дребезга $T_{bounce}$. Результаты эксперимента приведены в таблице \ref{tab:0to1} и на рис. \ref{fig:0to1}.

\begin{table}[H]
	\def\tabcolsep{20pt}
	\centering
	\caption{Исследование дребезга при переключениях из 0 => 1}
	\label{tab:0to1}
	\begin{tabular}{|c|c|c|c|c|c|c|}
		\hline
		№ & 1 & 2 & 3 & 4 & 5 & 6 \\ \hline
		$N_{stp}$ & 8 & 3 & 22 & 20 & 5 & 31 \\ \hline
		$N_{ctn}$ & 8 & 3 & 22 & 20 & 5 & 31 \\ \hline
		$T_{bounce}$, мкс & 48 & 10 & 40 & 68 & 5 & 72 \\ \hline
	\end{tabular}
\end{table}

\begin{figure}[H]
	\centering
	\begin{subfigure}[b]{\textwidth}
		\centering
		\includegraphics[width=0.9\textwidth]{0to1_1}
		\vspace{0.2cm}
	\end{subfigure}
	\begin{subfigure}[b]{\textwidth}
		\centering
		\includegraphics[width=0.9\textwidth]{0to1_2}
		\vspace{0.2cm}
	\end{subfigure}
	\begin{subfigure}[b]{\textwidth}
		\centering
		\includegraphics[width=0.9\textwidth]{0to1_3}
		\vspace{0.2cm}
	\end{subfigure}
	\begin{subfigure}[b]{\textwidth}
		\centering
		\includegraphics[width=0.9\textwidth]{0to1_4}
		\vspace{0.2cm}
	\end{subfigure}
	\begin{subfigure}[b]{\textwidth}
		\centering
		\includegraphics[width=0.9\textwidth]{0to1_5}
		\vspace{0.2cm}
	\end{subfigure}
	\begin{subfigure}[b]{\textwidth}
		\centering
		\includegraphics[width=0.9\textwidth]{0to1_6}
		\vspace{0.2cm}
	\end{subfigure}
	\caption{Исследование дребезга при переключениях из 0 => 1}
	\label{fig:0to1}
\end{figure}

Из результатов видно, что максимальное число переходов из 0 в 1 оказалось равно 31, а максимальная длительность 72 мкс.

\subsection{Исследование дребезга при переключениях из 1 => 0}

Переведем \code{sw[0]} в положение 1 и сбросим счетчик с помощью кнопки \code{pbb}. Проведем эксперимент, переключая \code{sw[0]} из 1 в 0, будем подсчитывать количество импульсов $N_{stp}$ на плате, отображаемых на светодиодах, количество импульсов $N_{cnt}$, подсчитанных с помощью логического анализатора ELA1 и длительность процесса дребезга $T_{bounce}$. Результаты эксперимента приведены в таблице \ref{tab:1to0} и на рис. \ref{fig:1to0}.

\begin{table}[H]
	\def\tabcolsep{20pt}
	\centering
	\caption{Исследование дребезга при переключениях из 1 => 0}
	\label{tab:1to0}
	\begin{tabular}{|c|c|c|c|c|c|c|}
		\hline
		№ & 1 & 2 & 3 & 4 & 5 & 6 \\ \hline
		$N_{stp}$ & 2 & 14 & 12 & 14 & 3 & 17 \\ \hline
		$N_{ctn}$ & 2 & 14 & 12 & 14 & 3 & 17 \\ \hline
		$T_{bounce}$, мкс & 44 & 76 & 40 & 13 & 36 & 92 \\ \hline
	\end{tabular}
\end{table}

\begin{figure}[H]
	\centering
	\begin{subfigure}[b]{\textwidth}
		\centering
		\includegraphics[width=0.9\textwidth]{1to0_1}
		\vspace{0.2cm}
	\end{subfigure}
	\begin{subfigure}[b]{\textwidth}
		\centering
		\includegraphics[width=0.9\textwidth]{1to0_2}
		\vspace{0.2cm}
	\end{subfigure}
	\begin{subfigure}[b]{\textwidth}
		\centering
		\includegraphics[width=0.9\textwidth]{1to0_3}
		\vspace{0.2cm}
	\end{subfigure}
	\begin{subfigure}[b]{\textwidth}
		\centering
		\includegraphics[width=0.9\textwidth]{1to0_4}
		\vspace{0.2cm}
	\end{subfigure}
	\begin{subfigure}[b]{\textwidth}
		\centering
		\includegraphics[width=0.9\textwidth]{1to0_5}
		\vspace{0.2cm}
	\end{subfigure}
	\begin{subfigure}[b]{\textwidth}
		\centering
		\includegraphics[width=0.9\textwidth]{1to0_6}
		\vspace{0.2cm}
	\end{subfigure}
	\caption{Исследование дребезга при переключениях из 1 => 0}
	\label{fig:1to0}
\end{figure}

Из результатов видно, что максимальное число переходов из 1 в 0 оказалось равно 17, а максимальная длительность 76 мкс.

\section{Реализация устройства защиты от дребезга}

Борьба с дребезгом контактов в схеме, изображенная на рис. \ref{fig:source_fixed}, осуществляется на базе счетчика \code{cnt_d[3..0]}, разрядность которого задана в индивидуальном задании. Так как весь процесс дребезга в пунктах \ref{sec:cb} полностью фиксировался в анализаторе, то счетчик-делителя в данном случае оказался не нужен. Счетчик с выходами \code{cnt_s[3..0]} осуществляет подсчет переключений \code{sw} после реализации мер борьбы с дребезгом контактов. Выводы \code{temp[3..0]} и \code{cnt_s[3..0]} -- временные выводы, обеспечивающие наблюдаемость схемы.

Алгоритм борьбы с дребезгом контактов:
\begin{enumerate}
	\vspace{-0.5em}
	\item Вводится тактовый домен.
	\item Сигнал с кнопки снимается выходным триггером, синхронизируемым тактовым доменом.
	\item Выход триггера регулирует работу специального счетчика (устанавливает направление счета), счетчик синхронизируется тем же тактовым доменом.
	\item При заполнении этого счетчика выход \code{cout} становится активным, запрещая счет (выход \code{cout} соединен через инвертор со входом \code{cnt_en}).
	\item Выход \code{cout} является также сигналом разрешения работы выходного триггера, синхронизируемого тем же тактовым доменом. Вход \code{D} выходного триггера соединен с выходом \code{Q} входного триггера.
\end{enumerate}

\begin{figure}[H]
	\begin{center}
		\includegraphics[scale=0.95]{source_fixed}
		\caption{Устройство защиты от дребезга}
		\label{fig:source_fixed}
	\end{center}
\end{figure}

\section{Выводы}

Для изучения процесса дребезга контактов использован логический анализатор SignalTapII. Исследовалась схема подсчета импульсов дребезга, анализировались случаи переключения 0 => 1 и 1 => 0. Затем была построена схема борьбы с дребезгом на основе счетчика с заданной в индивидуальном задании разрядностью. Исследование этой схемы с помощью логического анализатора позволило на практике увидеть процесс дребезга и работу алгоритма борьбы с ним.

\end{document}