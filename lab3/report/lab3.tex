\documentclass[a4paper,14pt]{extarticle}

\usepackage[utf8x]{inputenc}
\usepackage[T1,T2A]{fontenc}
\usepackage[russian]{babel}
\usepackage{hyperref}
\usepackage{indentfirst}
\usepackage{here}
\usepackage{array}
\usepackage{graphicx}
\usepackage{caption}
\usepackage{subcaption}
\usepackage{chngcntr}
\usepackage{amsmath}
\usepackage{amssymb}
\usepackage{pgfplots}
\usepackage{pgfplotstable}
\usepackage[left=2cm,right=2cm,top=2cm,bottom=2cm,bindingoffset=0cm]{geometry}
\usepackage{multicol}
\usepackage{askmaps}
\usepackage{titlesec}

\renewcommand{\le}{\ensuremath{\leqslant}}
\renewcommand{\leq}{\ensuremath{\leqslant}}
\renewcommand{\ge}{\ensuremath{\geqslant}}
\renewcommand{\geq}{\ensuremath{\geqslant}}
\renewcommand{\epsilon}{\ensuremath{\varepsilon}}
\renewcommand{\phi}{\ensuremath{\varphi}}
\renewcommand{\thefigure}{\arabic{figure}} 	
\renewcommand*\not[1]{\overline{#1}}

\titleformat*{\section}{\large\bfseries} 
\titleformat*{\subsection}{\normalsize\bfseries} 
\titleformat*{\subsubsection}{\normalsize\bfseries} 
\titleformat*{\paragraph}{\normalsize\bfseries} 
\titleformat*{\subparagraph}{\normalsize\bfseries} 

\counterwithin{figure}{section}
\counterwithin{equation}{section}
\counterwithin{table}{section}
\newcommand{\sign}[1][5cm]{\makebox[#1]{\hrulefill}}
\graphicspath{{../pics/}}
\captionsetup{justification=centering,margin=1cm}
\def\arraystretch{1.3}
\setlength\parindent{5ex}
\titlelabel{\thetitle.\quad}

\begin{document}

\begin{titlepage}
\begin{center}
	Санкт-Петербургский Политехнический Университет Петра Великого\\[3mm]
	Институт компьютерных наук и технологий \\[3mm]
	Кафедра компьютерных систем и программных технологий\\[6cm]
	
	\textbf{КУРСОВОЙ ПРОЕКТ}\\[3mm]
	\textbf{Разработка устройства передачи данных}\\[3mm]
	по дисциплине <<Автоматизация проектирования\\[3mm] 
	дискретных устройств>>\\[6cm]
\end{center}

\begin{flushleft}
	Выполнил\\[3mm]
	студент гр. 33501/4  \hspace*{7.1cm} Дьячков В.В.\\[3mm]
	Руководитель\\[3mm]
	доцент, к.т.н. \hspace*{8.5cm} А.С. Филиппов\\[4mm]
	\hspace*{11.5cm} <<\sign[7mm]>> \sign[20mm] \the\year\hspace{1mm} г.
\end{flushleft}

\vfill

\begin{center}
	Санкт-Петербург\\
	\the\year
\end{center}
\end{titlepage}

\addtocounter{page}{1}
\counterwithin{lstlisting}{section}

\tableofcontents
%\newpage
\listoffigures
%\lstlistoflistings
\newpage

\section{lab3\_1}

\subsection{Задание}

На языке Verilog опишите логическую функцию 5-ти переменных:
\begin{itemize}
	\item Функция:
		\begin{itemize}
			\item $f(X) = 1$, если $X \in \{0\text{-}2,5,8\text{-}9,11\text{-}14,19\text{-}21,24,26,29\text{-}30\}$;
			\item $f(X) =$\ H, если $X \in \{6,16,23,28\}$;
			\item $f(X) = 0$, иначе.
		\end{itemize}
	\item Входы: переключатели \code{sw[4:0]}.
	\item Выходы:
		\begin{itemize}
			\item Светодиод \code{led[0]} -- выход логической функции 5-ти переменных (если значение логической функции \code{= 1}, то светодиод должен быть включен);
			\item Светодиод \code{led[1]} -- выход \code{= 1} (светодиод включен) для всех незаданных наборов аргументов.
		\end{itemize}
\end{itemize}

В описании можно использовать операторы case, casex, bitwise. Тип данных - \code{wire}.

\subsection{Код на языке Verilog}

В листинге \ref{code:1} приведен код программы на языке Verilog.

\lstinputlisting[caption=lab3\_1.v, label=code:1]{lab3_1/lab3_1.v}

\subsection{Результаты синтеза}

На рис. \ref{fig:lab3_1_rtl} приведено изображение синтезированной схемы в RLT Viewer.

\begin{figure}[H]
\begin{center}
	\includegraphics[scale=0.9]{lab3_1_rtl}
	\caption{Результат синтеза в RLT Viewer}
	\label{fig:lab3_1_rtl}
\end{center}
\end{figure}

\subsection{Результаты моделирования}
\label{sec:lab3_1_modeling}

На рис. \ref{fig:lab3_1_modeling} изображена временная диаграмма работы синтезированного устройства. На вход устройству поочередно подаются всевозможные значения \code{x[4:0]}, значение функции записывается в \code{res}, а \code{x_state = 1} при незаданных наборах аргументов.

\begin{figure}[H]
\begin{center}
	\includegraphics[width=\textwidth]{lab3_1_modeling}
	\caption{Результаты моделирования}
	\label{fig:lab3_1_modeling}
\end{center}
\end{figure}

\subsection{Назначение выводов СБИС}

На рис. \ref{fig:lab3_1_pins} приведены назначения выводов СБИС в Pin Planner.

\begin{figure}[H]
\begin{center}
	\includegraphics{lab3_1_pins}
	\caption{Таблица назначений в Pin Planer}
	\label{fig:lab3_1_pins}
\end{center}
\end{figure}

\subsection{Результаты проверки на плате}

Для тестирования проекта на плате были использованы тесты, описанные в пункте \ref{sec:lab3_1_modeling}. Результаты тестирования совпадают с ожидаемыми, следовательно, устройство работает верно.

\subsection{Выводы}

Реализована логическая функция. В описании использовались операторы Concatenate и case. Результаты моделирования и тестирования на плате показали, что разработанное устройство работает верно.

\newpage

\section{lab3\_2}

\subsection{Задание}

На языке Verilog опишите преобразователь 4-х разрядного двоичного кода в 7-сегментный код, с управлением точкой:
\begin{itemize}
	\item Входы:
		\begin{itemize}
			\item переключатели \code{sw[3:0]} -- 4-х разрядный двоичный код;
			\item кнопка \code{pba} -- управление точкой на 7-сегментном индикаторе;
			\item кнопка \code{pbb} -- разрешение отображения
				\begin{itemize}
					\item[$\circ$] при \code{pbb = 1} отображение на 7-сегментном индикаторе разрешено;
					\item[$\circ$] при \code{pbb = 0} отображение на 7-сегментном индикаторе запрещено (все сегменты, включая точку, выключены).
				\end{itemize}
		\end{itemize}		
	\item Выходы: светодиоды \code{led[7:0]}.
\end{itemize}

В описании можно использовать операторы case, if, bitwise. Тип данных - \code{wire}. На рис. \ref{fig:lab3_2_segment} изображен 7-сегментный индикатор с управлением точкой.

\begin{figure}[H]
\begin{center}
	\begin{tikzpicture}
		\tikzstyle{every node}=[draw=none,text=black]
		\SSGNb{}{8};
		\node at (0, 1.9) {$1$};
		\node at (1.1, 0.8) {$2$};
		\node at (1.1, -0.7) {$3$};
		\node at (0, -1.9) {$4$};
		\node at (-1.1, -0.7) {$5$};
		\node at (-1.1, 0.8) {$6$};
		\node at (0, 0.4) {$7$};
		\node at (1.2, -1.9) {dot};
		\node[text=green, mark size=3pt] at (1.2,-1.4) {\pgfuseplotmark{square*}};
	\end{tikzpicture}
	\caption{7-сегментный индикатор с управлением точкой}
	\label{fig:lab3_2_segment}
\end{center}
\end{figure}

\newpage

\subsection{Код на языке Verilog}

В листинге \ref{code:2} приведен код программы на языке Verilog.

\lstinputlisting[caption=lab3\_2.v, label=code:2]{lab3_2/lab3_2.v}

\newpage

\subsection{Результаты синтеза}

На рис. \ref{fig:lab3_2_rtl} приведено изображение синтезированной схемы в RLT Viewer.

\begin{figure}[H]
\begin{center}
	\includegraphics[scale=1]{lab3_2_rtl}
	\caption{Результат синтеза в RLT Viewer}
	\label{fig:lab3_2_rtl}
\end{center}
\end{figure}

\newpage

\subsection{Результаты моделирования}
\label{sec:lab3_2_modeling}

На рис. \ref{fig:lab3_2_modeling} изображена временная диаграмма работы синтезированного устройства при рабочих входах. На вход устройству поочередно подаются значения от \code{0} до {9}, что соответствует рабочим входам устройства.

\begin{figure}[H]
\begin{center}
	\includegraphics[width=\textwidth]{lab3_2_modeling}
	\caption{Результаты моделирования }
	\label{fig:lab3_2_modeling}
\end{center}
\end{figure}

На рис. \ref{fig:lab3_2_modeling_2} изображена временная диаграмма работы синтезированного устройства при случайных входных значениях. Проверяется правильность работы управляющих сигналов \code{ena}, разрешающего отображение, и \code{dot_ena}, отвечающего за отображение точки.

\begin{figure}[H]
\begin{center}
	\includegraphics[width=\textwidth]{lab3_2_modeling_2}
	\caption{Результаты моделирования}
	\label{fig:lab3_2_modeling_2}
\end{center}
\end{figure}

\newpage

\subsection{Назначение выводов СБИС}

На рис. \ref{fig:lab3_2_pins} приведены назначения выводов СБИС в Pin Planner.

\begin{figure}[H]
\begin{center}
	\includegraphics{lab3_2_pins}
	\caption{Таблица назначений в Pin Planer}
	\label{fig:lab3_2_pins}
\end{center}
\end{figure}

\subsection{Результаты проверки на плате}

Для тестирования проекта на плате были использованы тесты, описанные в пункте \ref{sec:lab3_2_modeling}. Результаты тестирования совпадают с ожидаемыми, следовательно, устройство работает верно.

\subsection{Выводы}

Реализован преобразователь 4-х разрядного двоичного кода в 7-сегментный код, с управлением точкой. Результаты моделирования и тестирования на плате показали, что разработанное устройство работает верно.

\newpage

\section{lab3\_3}

\subsection{Задание}

На языке Verilog опишите устройство, сравнивающее два 4-х разрядных числа и передающее на выход большее из них:
\begin{itemize}
\item Входы:
	\begin{itemize}
		\item Первое число -- переключатели \code{sw[7:4]};
		\item Второе число -- переключатели \code{sw[3:0]}.
	\end{itemize}
\item Выходы: светодиоды \code{led[3:0]}.
\end{itemize}

В описании можно использовать операторы if, case, сравнения, Bitwise, Logical, Reduction, Reduction, Replicator, Concatenate. Тип данных - \code{wire}.

\subsection{Код на языке Verilog}

В листинге \ref{code:3} приведен код программы на языке Verilog.

\lstinputlisting[caption=lab3\_3.v, label=code:3]{lab3_3/lab3_3.v}

\subsection{Результаты синтеза}

На рис. \ref{fig:lab3_3_rtl} приведено изображение синтезированной схемы в RLT Viewer.

\begin{figure}[H]
\begin{center}
	\includegraphics[scale=0.8]{lab3_3_rtl}
	\caption{Результат синтеза в RLT Viewer}
	\label{fig:lab3_3_rtl}
\end{center}
\end{figure}

\subsection{Результаты моделирования}
\label{sec:lab3_3_modeling}

На рис. \ref{fig:lab3_3_modeling} изображена временная диаграмма работы синтезированного устройства. На вход поочередно подаются случайные значения \code{a[3:0]} и \code{b[3:0]}, больше из них записывается в \code{max[3:0]}.

\begin{figure}[H]
\begin{center}
	\includegraphics[width=\textwidth]{lab3_3_modeling}
	\caption{Результаты моделирования}
	\label{fig:lab3_3_modeling}
\end{center}
\end{figure}

\subsection{Назначение выводов СБИС}

На рис. \ref{fig:lab3_3_pins} приведены назначения выводов СБИС в Pin Planner.

\begin{figure}[H]
\begin{center}
	\includegraphics{lab3_3_pins}
	\caption{Таблица назначений в Pin Planer}
	\label{fig:lab3_3_pins}
\end{center}
\end{figure}

\subsection{Результаты проверки на плате}

Для тестирования проекта на плате были использованы тесты, описанные в пункте \ref{sec:lab3_3_modeling}. Результаты тестирования совпадают с ожидаемыми, следовательно, устройство работает верно.

\subsection{Выводы}

Реализовано устройство, сравнивающее два 4-х разрядных числа и передающее на выход большее из них. Результаты моделирования и тестирования на плате показали, что разработанное устройство работает верно.

\newpage

\section{lab3\_4}

\subsection{Задание}

На языке Verilog опишите мажоритарное устройство с тремя 2-х разрядными входами:
\begin{itemize}
	\item Алгоритм: устройство анализирует три 2-х разрядных входа
		\begin{itemize}
			\item Если любые два (или все три) входа имеют одинаковое значение, то оно передается на информационный выход устройства, при этом формируется признак достоверности данных
				\begin{itemize}
					\item[$\circ$] \code{= 2}, если значения любых двух входов были одинаковыми;
					\item[$\circ$] \code{= 3}, если значения трех входов были одинаковыми.
				\end{itemize}
			\item Если все входы имеют разное значение, то на информационном выходе устройства может формироваться произвольное значение, при этом формируется признак достоверности данных \code{= 0}.
		\end{itemize}
	\item Входы: переключатели \code{sw[5:4]}, \code{sw[3:2]}, \code{sw[1:0]} - для трех 2-х разрядных чисел.
	\item Выходы:
		\begin{itemize}
			\item светодиоды \code{led[1:0]} -- информационные выходы;
			\item светодиоды \code{led[3:2]} -- выходы признака достоверности данных.
		\end{itemize}
\end{itemize}

Необходимо обеспечить включение светодиодов логической единицей. В описании можно использовать любые операторы if, case, сравнения Bitwise, Logical, Reduction, Reduction, Replicator, Concatenate. Тип данных - \code{wire}.

\subsection{Код на языке Verilog}

В листинге \ref{code:4} приведен код программы на языке Verilog.

\lstinputlisting[caption=lab3\_4.v, label=code:4]{lab3_4/lab3_4.v}

\subsection{Результаты синтеза}

На рис. \ref{fig:lab3_4_rtl} приведено изображение синтезированной схемы в RLT Viewer.

\begin{figure}[H]
\begin{center}
	\includegraphics[width=\textwidth]{lab3_4_rtl}
	\caption{Результат синтеза в RLT Viewer}
	\label{fig:lab3_4_rtl}
\end{center}
\end{figure}

\subsection{Результаты моделирования}
\label{sec:lab3_4_modeling}

На рис. \ref{fig:lab3_4_modeling} изображена временная диаграмма работы синтезированного устройства. На вход устройства поочередно подаются случайные значения \code{a[1:0]}, \code{b[1:0]} и \code{c[1:0]}, признаки достоверности записываются в \code{flags[1:0]}, а мажоритарное значение (если значения хотя бы двух входов одинаковы) в \code{major[1:0]}.

\begin{figure}[H]
\begin{center}
	\includegraphics[width=\textwidth]{lab3_4_modeling}
	\caption{Результаты моделирования}
	\label{fig:lab3_4_modeling}
\end{center}
\end{figure}

\subsection{Назначение выводов СБИС}

На рис. \ref{fig:lab3_4_pins} приведены назначения выводов СБИС в Pin Planner.

\begin{figure}[H]
\begin{center}
	\includegraphics{lab3_4_pins}
	\caption{Таблица назначений в Pin Planer}
	\label{fig:lab3_4_pins}
\end{center}
\end{figure}

\subsection{Результаты проверки на плате}

Для тестирования проекта на плате были использованы тесты, описанные в пункте \ref{sec:lab3_4_modeling}. Результаты тестирования совпадают с ожидаемыми, следовательно, устройство работает верно.

\subsection{Выводы}

Реализовано мажоритарное устройство с тремя 2-х разрядными входами. Результаты моделирования и тестирования на плате показали, что разработанное устройство работает верно.

\section{elab3\_1}

\subsection{Задание}

На языке Verilog опишите преобразователь 4-х разрядного двоичного кода в код Грея:
\begin{itemize}
	\item Входы: переключатели \code{sw[3:0]} -- 4-х разрядный двоичный код.
	\item Выходы: светодиоды \code{led[3:0]} –- код Грея.
\end{itemize}

В описании можно использовать операторы Bitwise, Logical, Reduction, Reduction, Replicator, Concatenate. Тип данных -- \code{wire}.

\subsection{Код на языке Verilog}

В листинге \ref{code:5} приведен код программы на языке Verilog.

\lstinputlisting[caption=elab3\_1.v, label=code:5]{elab3_1/elab3_1.v}

\subsection{Результаты синтеза}

На рис. \ref{fig:elab3_1_rtl} приведено изображение синтезированной схемы в RLT Viewer.

\begin{figure}[H]
\begin{center}
	\includegraphics[scale=0.9]{elab3_1_rtl}
	\caption{Результат синтеза в RLT Viewer}
	\label{fig:elab3_1_rtl}
\end{center}
\end{figure}

\subsection{Результаты моделирования}
\label{sec:elab3_1_modeling}

На рис. \ref{fig:elab3_1_modeling} изображена временная диаграмма работы синтезированного устройства. На вход устройства поочередно подаются всевозможные значения \code{binary[3:0]}, соответствующие числам от \code{0} до \code{15}, а преобразованное значение в код Грея записывается в \code{gray[3:0]}.

\begin{figure}[H]
\begin{center}
	\includegraphics[width=\textwidth]{elab3_1_modeling}
	\caption{Результаты моделирования}
	\label{fig:elab3_1_modeling}
\end{center}
\end{figure}

\subsection{Назначение выводов СБИС}

На рис. \ref{fig:elab3_1_pins} приведены назначения выводов СБИС в Pin Planner.

\begin{figure}[H]
\begin{center}
	\includegraphics{elab3_1_pins}
	\caption{Таблица назначений в Pin Planer}
	\label{fig:elab3_1_pins}
\end{center}
\end{figure}

\subsection{Результаты проверки на плате}

Для тестирования проекта на плате были использованы тесты, описанные в пункте \ref{sec:elab3_1_modeling}. Результаты тестирования совпадают с ожидаемыми, следовательно, устройство работает верно.

\subsection{Выводы}

Реализован преобразователь 4-х разрядного двоичного кода в код Грея. Результаты моделирования и тестирования на плате показали, что разработанное устройство работает верно.

\section{elab3\_2}

\subsection{Задание}

На языке Verilog опишите преобразователь 4-х разрядного кода Грея в двоичный код:
\begin{itemize}
	\item Входы: переключатели \code{sw[3:0]} -- 4-х разрядный код Грея.
	\item Выходы: светодиоды \code{led[3:0]} -– двоичный код.
\end{itemize}

В описании можно использовать операторы Bitwise, Logical, Reduction, Reduction, Replicator, Concatenate. Тип данных - \code{wire}.

\subsection{Код на языке Verilog}

В листинге \ref{code:6} приведен код программы на языке Verilog.

\lstinputlisting[caption=elab3\_2.v, label=code:6]{elab3_2/elab3_2.v}

\subsection{Результаты синтеза}

На рис. \ref{fig:elab3_2_rtl} приведено изображение синтезированной схемы в RLT Viewer.

\begin{figure}[H]
\begin{center}
	\includegraphics[scale=0.9]{elab3_2_rtl}
	\caption{Результат синтеза в RLT Viewer}
	\label{fig:elab3_2_rtl}
\end{center}
\end{figure}

\subsection{Результаты моделирования}
\label{sec:elab3_2_modeling}

На рис. \ref{fig:elab3_2_modeling} изображена временная диаграмма работы синтезированного устройства. На вход устройства поочередно подаются всевозможные значения \code{gray[3:0]}, соответствующие числам от \code{0} до \code{15}, а преобразованное значение в код Грея записывается в \code{binary[3:0]}.

\begin{figure}[H]
\begin{center}
	\includegraphics[width=\textwidth]{elab3_2_modeling}
	\caption{Результаты моделирования}
	\label{fig:elab3_2_modeling}
\end{center}
\end{figure}

\subsection{Назначение выводов СБИС}

На рис. \ref{fig:elab3_2_pins} приведены назначения выводов СБИС в Pin Planner.

\begin{figure}[H]
\begin{center}
	\includegraphics{elab3_2_pins}
	\caption{Таблица назначений в Pin Planer}
	\label{fig:elab3_2_pins}
\end{center}
\end{figure}

\subsection{Результаты проверки на плате}

Для тестирования проекта на плате были использованы тесты, описанные в пункте \ref{sec:elab3_2_modeling}. Результаты тестирования совпадают с ожидаемыми, следовательно, устройство работает верно.

\subsection{Выводы}

Реализован преобразователь 4-х разрядного кода Грея в двоичный код. Результаты моделирования и тестирования на плате показали, что разработанное устройство работает верно.

\end{document}