\documentclass[a4paper,14pt]{extarticle}

\usepackage[utf8x]{inputenc}
\usepackage[T1,T2A]{fontenc}
\usepackage[russian]{babel}
\usepackage{hyperref}
\usepackage{indentfirst}
\usepackage{here}
\usepackage{array}
\usepackage{graphicx}
\usepackage{caption}
\usepackage{subcaption}
\usepackage{chngcntr}
\usepackage{amsmath}
\usepackage{amssymb}
\usepackage{pgfplots}
\usepackage{pgfplotstable}
\usepackage[left=2cm,right=2cm,top=2cm,bottom=2cm,bindingoffset=0cm]{geometry}
\usepackage{multicol}
\usepackage{askmaps}
\usepackage{titlesec}

\renewcommand{\le}{\ensuremath{\leqslant}}
\renewcommand{\leq}{\ensuremath{\leqslant}}
\renewcommand{\ge}{\ensuremath{\geqslant}}
\renewcommand{\geq}{\ensuremath{\geqslant}}
\renewcommand{\epsilon}{\ensuremath{\varepsilon}}
\renewcommand{\phi}{\ensuremath{\varphi}}
\renewcommand{\thefigure}{\arabic{figure}} 	
\renewcommand*\not[1]{\overline{#1}}

\titleformat*{\section}{\large\bfseries} 
\titleformat*{\subsection}{\normalsize\bfseries} 
\titleformat*{\subsubsection}{\normalsize\bfseries} 
\titleformat*{\paragraph}{\normalsize\bfseries} 
\titleformat*{\subparagraph}{\normalsize\bfseries} 

\counterwithin{figure}{section}
\counterwithin{equation}{section}
\counterwithin{table}{section}
\newcommand{\sign}[1][5cm]{\makebox[#1]{\hrulefill}}
\graphicspath{{../pics/}}
\captionsetup{justification=centering,margin=1cm}
\def\arraystretch{1.3}
\setlength\parindent{5ex}
\titlelabel{\thetitle.\quad}

\begin{document}

\begin{titlepage}
\begin{center}
	Санкт-Петербургский Политехнический Университет Петра Великого\\[3mm]
	Институт компьютерных наук и технологий \\[3mm]
	Кафедра компьютерных систем и программных технологий\\[6cm]
	
	\textbf{КУРСОВОЙ ПРОЕКТ}\\[3mm]
	\textbf{Разработка устройства передачи данных}\\[3mm]
	по дисциплине <<Автоматизация проектирования\\[3mm] 
	дискретных устройств>>\\[6cm]
\end{center}

\begin{flushleft}
	Выполнил\\[3mm]
	студент гр. 33501/4  \hspace*{7.1cm} Дьячков В.В.\\[3mm]
	Руководитель\\[3mm]
	доцент, к.т.н. \hspace*{8.5cm} А.С. Филиппов\\[4mm]
	\hspace*{11.5cm} <<\sign[7mm]>> \sign[20mm] \the\year\hspace{1mm} г.
\end{flushleft}

\vfill

\begin{center}
	Санкт-Петербург\\
	\the\year
\end{center}
\end{titlepage}

\addtocounter{page}{1}
\counterwithin{lstlisting}{section}

%\tableofcontents
%\newpage
%\listoffigures
%\lstlistoflistings
\newpage

\section{Задание}

\begin{itemize}
	\item Понимание причин возникновения сбоев, вызванных метастабильностью, и способов привязки асинхронных сигналов в Quartus Prime;
	\item Получение навыков анализа метастабильности с использованием TimeQuest Timing Analyzer.
\end{itemize}

\section{Выполнение работы}

\subsection{Компиляция и анализ проекта, задание временных требований}

В подготовленном заранее проекте \code{Metastabil}, приведенном на рис. \ref{fig:source}, не выполняются требования к времени установки.
\begin{figure}[H]
	\begin{center}
		\includegraphics[width=\textwidth]{source}
		\caption{Исходный проект}
		\label{fig:source}
	\end{center}
\end{figure}

Зададим временные ограничения и режим анализа цепей синхронизации как \code{Forced If Asynchronous}. В листинге \ref{code:sdc} приведен SDC-файл.

\vspace{0.5cm}
\lstinputlisting[caption=SDC1.sdc, label=code:sdc]{SDC1.sdc}

\subsection{Компиляция и анализ проекта с заданными временными требованиями и анализ результатов}

После компиляции можно заметить, что в проекте используется один заданный базовый тактовый сигнал с частотой 100 МГц. и два порожденных тактовых сигнала с заданными при настройке PLL частотами.
\begin{figure}[H]
	\begin{center}
		\includegraphics[width=\textwidth]{clocks_1}
		\caption{Тактовые сигналы}
		\label{fig:clocks_1}
	\end{center}
\end{figure}

В разделе Slow 1200mV 85C Model Setup Summary, приведенном на рис. \ref{fig:timing_1}, видно, что временные требования не выполняются. Однако для остальных моделей ошибок нет, следовательно может оцениваться метастабильность.

\begin{figure}[H]
	\begin{center}
		\includegraphics[scale=0.95]{timing_1}
		\caption{Slow 1200mV 85C Model Setup Summary}
		\label{fig:timing_1}
	\end{center}
\end{figure}

На рис. \ref{fig:paths_1} приведен отчет Report Exceptions в TimeQuest Timing Analyzer. Видно, что заданные пути исключены из анализа. 

\begin{figure}[H]
	\begin{center}
		\includegraphics[width=\textwidth]{paths_1}
		\caption{Slow 1200mV 85C Model Setup Summary}
		\label{fig:paths_1}
	\end{center}
\end{figure}

На рис. \ref{fig:waveform_1} изображена временная диаграмма для регистровой передачи \code{From Node RG_in To Node RG_out}. Тактовым сигналом, запускающим данные в данной регистровой передаче является сигнал с вывода \code{clk[0] PLL_CLK} (Launch Clock), а захватывающим - сигнал с вывода \code{clk[1] PLL_CLK} (Latch Clock). Временной интервал между ними соответствует фазовому сдвигу, заданному при настройке \code{PLL_CLK}.

\begin{figure}[H]
	\begin{center}
		\includegraphics[width=\textwidth]{waveform_1}
		\caption{Slow 1200mV 85C Model Setup Summary}
		\label{fig:waveform_1}
	\end{center}
\end{figure}

Для обеспечения работоспособности цепи увеличим фазовый сдвиг на время отрицательного допуска Setup, равное 58 пс. Из отчета компиляции, приведенного на рис. \ref{fig:timing_2}, видно, что теперь временные требования выполнены. 

\begin{figure}[H]
	\begin{center}
		\includegraphics[scale=0.95]{timing_2}
		\caption{Slow 1200mV 85C Model Setup Summary}
		\label{fig:timing_2}
	\end{center}
\end{figure}

Полученный проект не содержит временных ошибок и для него может проводиться анализ метастабильности для всех моделей задержек. Установлен минимальный положительный допуск Setup, для которого интенсивность сбоев, вызванных метастабильностью, будет максимальной.

\subsection{Анализ интенсивности сбоев, вызванных метастабильностью}

На рис. \ref{fig:assignments_1} приведена заданная активность переключений в Assignment Editor, равная 20 МГц.

\begin{figure}[H]
	\begin{center}
		\includegraphics[scale=0.95]{assignments_1}
		\caption{Активность переключений в Assignment Editor}
		\label{fig:assignments_1}
	\end{center}
\end{figure}

На рис. \ref{fig:timing_2} приведен Metastability Report для модели Slow 1200mV 85C Model Setup. Из отчета видно, что частота переключений Data Toggle Rate, используемая для вычисления Mean Time Before Faille (MTBF), соответствует заданной.

\begin{figure}[H]
	\begin{center}
		\includegraphics[scale=0.95]{timing_3}
		\caption{Metastability Report для Slow 1200mV 85C Model}
		\label{fig:timing_2}
	\end{center}
\end{figure}

В таблице \ref{tab:research} приведены результаты исследования для разных моделей и значений сдвига фазы \code{clk[1]}.

\begin{table}[H]
	\centering
	\caption{Исследование метастабильности}
	\label{tab:research}
	\begin{tabular}{|c|p{0.2\linewidth}|p{0.1\linewidth}|p{0.1\linewidth}|p{0.1\linewidth}|p{0.1\linewidth}|p{0.1\linewidth}|p{0.1\linewidth}|}
		\hline
		\multirow{2}{*}{\makecell{\\ \\ №}} & \multirow{2}{*}{\makecell{\\ \\ Сдвиг фазы / \\ Output slack, нс}} & \multicolumn{2}{l|}{\makecell{\centering Slow 85C}} & \multicolumn{2}{l|}{\makecell{\centering Slow 0C}} & \multicolumn{2}{l|}{\makecell{\centering Fast 0C}} \\ \cline{3-8} 
		&  & Available Setting Time, нс & Typical MTBF, лет & Available Setting Time, нс & Typical MTBF, лет & Available Setting Time, нс & Typical MTBF, лет \\ 
		\hline
		1 &  &  &  &  &  &  &  \\ 
		\hline
		2 &  &  &  &  &  &  &  \\ 
		\hline
	\end{tabular}
\end{table}

\section{Выводы}

\end{document}