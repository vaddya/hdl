\documentclass[a4paper,14pt]{extarticle}

\usepackage[utf8x]{inputenc}
\usepackage[T1,T2A]{fontenc}
\usepackage[russian]{babel}
\usepackage{hyperref}
\usepackage{indentfirst}
\usepackage{here}
\usepackage{array}
\usepackage{graphicx}
\usepackage{caption}
\usepackage{subcaption}
\usepackage{chngcntr}
\usepackage{amsmath}
\usepackage{amssymb}
\usepackage{pgfplots}
\usepackage{pgfplotstable}
\usepackage[left=2cm,right=2cm,top=2cm,bottom=2cm,bindingoffset=0cm]{geometry}
\usepackage{multicol}
\usepackage{askmaps}
\usepackage{titlesec}

\renewcommand{\le}{\ensuremath{\leqslant}}
\renewcommand{\leq}{\ensuremath{\leqslant}}
\renewcommand{\ge}{\ensuremath{\geqslant}}
\renewcommand{\geq}{\ensuremath{\geqslant}}
\renewcommand{\epsilon}{\ensuremath{\varepsilon}}
\renewcommand{\phi}{\ensuremath{\varphi}}
\renewcommand{\thefigure}{\arabic{figure}} 	
\renewcommand*\not[1]{\overline{#1}}

\titleformat*{\section}{\large\bfseries} 
\titleformat*{\subsection}{\normalsize\bfseries} 
\titleformat*{\subsubsection}{\normalsize\bfseries} 
\titleformat*{\paragraph}{\normalsize\bfseries} 
\titleformat*{\subparagraph}{\normalsize\bfseries} 

\counterwithin{figure}{section}
\counterwithin{equation}{section}
\counterwithin{table}{section}
\newcommand{\sign}[1][5cm]{\makebox[#1]{\hrulefill}}
\graphicspath{{../pics/}}
\captionsetup{justification=centering,margin=1cm}
\def\arraystretch{1.3}
\setlength\parindent{5ex}
\titlelabel{\thetitle.\quad}

\begin{document}

\begin{titlepage}
\begin{center}
	Санкт-Петербургский Политехнический Университет Петра Великого\\[3mm]
	Институт компьютерных наук и технологий \\[3mm]
	Кафедра компьютерных систем и программных технологий\\[6cm]
	
	\textbf{КУРСОВОЙ ПРОЕКТ}\\[3mm]
	\textbf{Разработка устройства передачи данных}\\[3mm]
	по дисциплине <<Автоматизация проектирования\\[3mm] 
	дискретных устройств>>\\[6cm]
\end{center}

\begin{flushleft}
	Выполнил\\[3mm]
	студент гр. 33501/4  \hspace*{7.1cm} Дьячков В.В.\\[3mm]
	Руководитель\\[3mm]
	доцент, к.т.н. \hspace*{8.5cm} А.С. Филиппов\\[4mm]
	\hspace*{11.5cm} <<\sign[7mm]>> \sign[20mm] \the\year\hspace{1mm} г.
\end{flushleft}

\vfill

\begin{center}
	Санкт-Петербург\\
	\the\year
\end{center}
\end{titlepage}

\addtocounter{page}{1}
\counterwithin{lstlisting}{section}

\tableofcontents
\listoffigures
\lstlistoflistings
\newpage

\section{Задание}

На языке SystemVerilog опишите устройство, включающее:
\begin{itemize}
	\item Счетчик-делитель, обеспечивающий счет по модулю \code{25 000 000} и формирующий синхронный сигнал переноса (активный уровень сигнала \code{= 1}, длительность один такт тактовой частоты) по достижению счетчиком значения \code{25 000 000 - 1}.
	\item Двоичный, 4-разрядный, счетчик, алгоритм работы, которого задан приведенной ниже таблицей.
\vspace{-0.5cm}
\begin{table}[H]
\begin{center}
	\def\tabcolsep{13pt}
	\caption{Алгоритм работы счетчика}
	\begin{tabular}{|c|c|c|c|c|c|c|}
	\hline	
	\code{aset} & \code{ena} & \code{sclr} & \code{load} & \code{dir} & \code{din} & \code{q} \\ 
	\hline
	\code{0} & \code{X} & \code{X} & \code{X} & \code{X} & \code{X} & Асинхронная установка \\
	\hline
	\code{1} & \code{0} & \code{X} & \code{X} & \code{X} & \code{X} & Хранение \\
	\hline
	\code{1} & \code{1} & \code{0} & \code{X} & \code{X} & \code{X} & Синхронный сброс в \code{0} \\
	\hline
	\code{1} & \code{1} & \code{1} & \code{1} & \code{X} & \code{din} & Запись \code{din} \\
	\hline
	\code{1} & \code{1} & \code{1} & \code{0} & \code{1} & \code{X} & Счет \code{+} \\
	\hline
	\code{1} & \code{1} & \code{1} & \code{0} & \code{0} & \code{X} & Счет \code{-} \\
	\hline
	\end{tabular}
\end{center}
\end{table}	
\vspace{-0.5cm}
	\item Входы:
		\begin{itemize}
			\item Кнопка \code{pba} -- асинхронная установка (установка при нажатии на кнопку). Соединена с входом \code{aset} 4-разрядного счетчика.
			\item Кнопка \code{pbb} -- синхронный сброс (сброс при нажатии на кнопку). Соединена с входом \code{sclr} 4-разрядного счетчика.
			\item Переключатель \code{sw[1]} -- управление загрузкой счетчика. Соединен с входом \code{load} 4-разрядного счетчика.
			\item Переключатель \code{sw[0]} -- управление направлением счета. Соединен с входом \code{dir} 4-разрядного счетчика.
			\item Переключатели \code{sw[7:4]} -- загружаемые данные. Соединены с входом \code{din} 4-разрядного счетчика.
			\item Тактовый сигнал (\code{clk}) подается от тактового генератора (см. описание стенда). Частота тактового сигнала – 25МГц.
		\end{itemize}
	\item Выходы: светодиоды \code{led[7:4]} (двоичного 4-разрядного счетчика).
\end{itemize}

\section{Код на языке SystemVerilog}

В листинге \ref{code:1} приведен код программы на языке Verilog.

\lstinputlisting[caption=lab4\_1.sv, label=code:1]{lab4_1.sv}

\newpage

\section{Результаты синтеза}

На рис. \ref{fig:lab4_1_rtl} приведено изображение синтезированной схемы в RLT Viewer.

\begin{figure}[H]
\begin{center}
	\includegraphics[width=\textwidth]{lab4_1_rtl}
	\caption{Результат синтеза в RLT Viewer}
	\label{fig:lab4_1_rtl}
\end{center}
\end{figure}

\section{Результаты моделирования}
\label{sec:lab4_1_modeling}

На рис. \ref{fig:lab4_1_modeling} изображена временная диаграмма работы синтезированного устройства.

\begin{figure}[H]
\begin{center}
	\includegraphics[width=\textwidth]{lab4_1_modeling}
	\caption{Результаты моделирования}
	\label{fig:lab4_1_modeling}
\end{center}
\end{figure}

\newpage

\section{Назначение выводов СБИС}

На рис. \ref{fig:lab4_1_pins} приведены назначения выводов СБИС в Pin Planner.

\begin{figure}[H]
\begin{center}
	\includegraphics{lab4_1_pins}
	\caption{Таблица назначений в Pin Planer}
	\label{fig:lab4_1_pins}
\end{center}
\end{figure}

\section{Результаты проверки на плате}

Для тестирования проекта на плате были использованы тесты, описанные в пункте \ref{sec:lab4_1_modeling}. Результаты тестирования совпадают с ожидаемыми, следовательно, устройство работает верно.

\section{Выводы}

Реализовано устройство, содержащее счетчик-делитель и двоичный 4-разрядный счетчик с заданным алгоритмом работы. В описании использовались операторы Concatenate и case. Результаты моделирования и тестирования на плате показали, что разработанное устройство работает верно.

\end{document}